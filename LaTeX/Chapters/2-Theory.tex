\chapter{Theoretical Framework}
\label{ch:theory}

1. Why Monte Carlo Can Struggle With Rare Events
When you draw samples from the original (or “nominal”) distribution—e.g., 
$N(\mu, \sigma^2)$ for each parameter—the probability of seeing “extreme” samples that cause failure (like crossing 
$y < -1$) may be very small. Hence, most samples do not contribute to the failure count, and the probability estimate can have a high variance (you need a huge number of samples to accurately capture rare failures).

Importance Sampling (IS) addresses this by sampling from a modified distribution that focuses on the region more likely to produce failure (i.e., “importance region”). Then each sample is re-weighted to maintain an unbiased estimate of the true probability.


You want to model:

$p(y,t)$ where the randomness comes from $(m,\mu,k) \sim$ known distribution
p(y,t) where the randomness comes from $(m,\mu,k) \sim$ known distribution
This is called a parametric uncertainty propagation problem. You have two main options: