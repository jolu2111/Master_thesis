\chapter{Introduction}



\section{Background and motivation}
The rice of deep learning has been a game changer in the field of machine learning. Deep learning has been used to solve a wide range of problems, such as image recognition, speech recognition, and natural language processing. 

However, deep learning models are often considered as black boxes, as they are difficult to interpret. This is a problem, as it is important to understand how a model makes predictions in order to trust it. In addition, deep learning models are often overconfident in their predictions, which can lead to catastrophic failures in safety-critical applications. Therefore, there is a need for methods that can provide uncertainty estimates for deep learning models.

In the field of engineering, uncertainty estimation is important for decision-making under uncertainty. For example, geotechnical structure design and analysis are affected by measurement uncertainty, statistical uncertainty, and transformation model uncertainty. These geotechnical uncertainties are not capable of being taken into account by conventional deterministic analytical methods. First of all, there are many different causes of uncertainty, including variations in soil characteristics, building techniques, and environmental factors. Therefore, there is a need for methods that can provide uncertainty estimates for geotechnical structure design and analysis.

There has been very little research on uncertainty estimation for deep learning models. One approach is to use Bayesian neural networks, which can provide uncertainty estimates by placing a distribution over the weights of a neural network. However, Bayesian neural networks are computationally expensive and difficult to train. Another approach is to use ensemble methods, which can provide uncertainty estimates by training multiple models on different subsets of the data. However, ensemble methods are also computationally expensive and difficult to train.

Physics infromed neural networks (PINNs) are a class of deep learning models that can be used to solve partial differential equations (PDEs). PINNs have been used to solve a wide range of PDEs, including the heat equation, the wave equation, and the Navier-Stokes equations. However, PINNs are often overconfident in their predictions, which can lead to catastrophic failures in safety-critical applications. Therefore, there is a need for methods that can provide uncertainty estimates for PINNs.

\section{Research question/objective} 
In this thesis, we use a metamodel-based importance sampling (MAIS) with a PINN as our metamodel. The objective of this thesis is to develop a method for uncertainty estimation for PINNs using MAIS. The research question is: Can we use MAIS with a PINN as our metamodel to provide uncertainty estimates for PINNs?


\section{Scope and limitations}

The thesis is limited to the use of MAIS with a PINN as our metamodel for uncertainty estimation. We do not consider other methods for uncertainty estimation, such as Bayesian neural networks or ensemble methods. We also do not consider other types of metamodels, such as Gaussian processes or radial basis functions. Furthermore, we focus on the use of MAIS with a PINN as our metamodel for uncertainty estimation. The partial differential equations considered are SO and SO

Due to 

\section{Thesis structure}



