\chapter{Methodology}
\label{ch:methodology}


\section*{Intuition behind multiplying $\pi(u)$ and $f(u)$:}

The algorithm you're using is essentially combining both \textit{prior knowledge} (from the CDF) and \textit{likelihood} (from the PDF) to form a \textit{combined target function}. The product $\pi(u) \cdot f(u)$ gives you a distribution that is \textbf{both cumulative and likelihood-based}.

\begin{itemize}
    \item \(\pi(u)\) captures the idea of how likely we are to be in a region up to \(u\). In the context of failure, if we think of \(g(\mathbf{u})\) as the performance measure, \(\pi(u)\) reflects the likelihood of reaching or exceeding the failure threshold, based on the \textit{cumulative distribution}.
    
    \item \(f(u)\) reflects how likely it is to actually land on a specific value of \(u\) (from the \textit{probability density}).
\end{itemize}

Together, multiplying the \textit{CDF} and the \textit{PDF} allows you to combine:
\begin{itemize}
    \item The likelihood of being in the \textit{failure region} (via \(\pi(u)\)),
    \item And the likelihood of specific configurations (via \(f(u)\)).
\end{itemize}

This combined term $\hat{h}(\mathbf{u})$ is what you're sampling from in the Metropolis-Hastings algorithm.

\textbf{Why multiply, not add?}
\begin{itemize}
    \item \textit{Multiplying} the two terms gives you a \textbf{joint distribution} where both the \textit{likelihood} of the parameters and the \textit{probability of failure} are considered together.
    \item If you were to \textit{add} $\pi(u)$ and $f(u)$, you would be combining two very different types of quantities (one representing cumulative probability, the other representing point likelihood), which would not give you a valid combined distribution.
\end{itemize}

\textbf{Conclusion:}

It is correct and \textit{intuitive} to multiply the CDF ($\pi(u)$) and the PDF ($f(u)$) in this case, because:
\begin{itemize}
    \item You are combining the \textit{probability of being in the failure region} (captured by $\pi(u)$),
    \item And the \textit{likelihood of a particular sample} (captured by $f(u)$).
\end{itemize}

This forms a valid joint distribution that you can use in the Metropolis-Hastings algorithm.



% How the paper (Importance Sampling for PINNs) actually chooses points After identifying interesting regions (via DWT or another metric), the actual point selection is done using a weighted probability distribution over the sample pool: “We compute the sampling probability for each point using a normalized score... and sample points with probability proportional to that score.” Importance sampling for… In other words: Start with a pool of points (e.g. from LHS or uniformly sampled). Assign each point a score, like: Residual (if training a PINN), Gradient magnitude, Or your custom criterion (e.g. 𝜙(𝑥)=exp(−𝛼∣𝑔(𝑥)∣), to favor points near 𝑔=0).


%------------------------------------------------------------
%  Interface conditions for layered Terzaghi consolidation
%------------------------------------------------------------
\paragraph{Interface conditions for a multi--layer Terzaghi problem}
Consider one--dimensional vertical consolidation of an $n$--layer soil column.
Within the $i$--th layer the excess pore pressure $u(t,z)$ satisfies
\begin{equation}
  \frac{\partial u}{\partial t}
  \;=\;
  c_{v,i}\,
  \frac{\partial^{2}u}{\partial z^{2}},
  \qquad
  c_{v,i}
  \;=\;
  \frac{k_i}{m_{v,i}\,\gamma_w},
  \label{eq:terzaghi_dim}
\end{equation}
where $k_i$ is the hydraulic conductivity, $m_{v,i}$ the coefficient of
volume compressibility and $\gamma_w$ the unit weight of water.
Let $z=z_I$ denote the interface between layer $i$ (``up") and layer $i{+}1$
(``down").  Two physical requirements must hold there:

\begin{enumerate}\setlength\itemsep{0pt}
\item \textbf{Continuity of pressure}
      \[
        u_i(t,z_I)=u_{i+1}(t,z_I),
      \]
\item \textbf{Continuity of Darcy flux}
      (mass conservation across the interface).
\end{enumerate}

The vertical Darcy flux is
\[
  q \;=\;
  -\,\frac{k}{\gamma_w}\,
  \frac{\partial u}{\partial z},
\]
hence equating $q$ above and below $z_I$ (and cancelling $1/\gamma_w$) yields
the flux--matching condition
\begin{equation}
  k_i\,
  \frac{\partial u_i}{\partial z}
  \;=\;
  k_{i+1}\,
  \frac{\partial u_{i+1}}{\partial z},
  \qquad z=z_I.
  \label{eq:flux_match_dim}
\end{equation}

\paragraph{Non--dimensional form}
Introduce scaled variables $z = H\,\tilde z$ and $t = T\,\tilde t$,
with $H=\sum_{j=1}^{n}H_j$ the total column height and $T$ a reference time
(e.g.\ the observation period).
Equation~\eqref{eq:terzaghi_dim} becomes
\[
  \frac{\partial u}{\partial\tilde t}
  \;=\;
  \alpha_i\,
  \frac{\partial^{2}u}{\partial\tilde z^{2}},
  \qquad
  \alpha_i
  \;=\;
  \frac{c_{v,i}\,T}{H^{2}},
\]
while the Darcy flux rewrites to
$
  q = -\,\dfrac{k}{\gamma_w H}\,\dfrac{\partial u}{\partial\tilde z}.
$
Because the prefactor $1/(\gamma_w H)$ is common to both layers it cancels,
so the dimensional condition~\eqref{eq:flux_match_dim} remains unchanged:
\[
  k_i\,
  u_{\tilde z,i}
  \;=\;
  k_{i+1}\,
  u_{\tilde z,i+1},
  \qquad z=\tilde z_I.
\]

\noindent
Therefore, irrespective of any space--time normalisation used in a
physics--informed neural network or finite--difference discretisation, the
correct interface conditions are
\begin{align*}
  u_i &= u_{i+1},\\[2pt]
  k_i\,u_z &= k_{i+1}\,u_z,
\end{align*}
while the storage parameter $m_v$ (and thus $c_v$ or $\alpha$) influences only
the intra--layer governing equation and plays \emph{no role} in the inter--layer
flux balance.
\