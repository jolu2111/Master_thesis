\chapter{Methodology}
\label{ch:methodology}


\section*{Intuition behind multiplying $\pi(u)$ and $f(u)$:}

The algorithm you're using is essentially combining both \textit{prior knowledge} (from the CDF) and \textit{likelihood} (from the PDF) to form a \textit{combined target function}. The product $\pi(u) \cdot f(u)$ gives you a distribution that is \textbf{both cumulative and likelihood-based}.

\begin{itemize}
    \item \(\pi(u)\) captures the idea of how likely we are to be in a region up to \(u\). In the context of failure, if we think of \(g(\mathbf{u})\) as the performance measure, \(\pi(u)\) reflects the likelihood of reaching or exceeding the failure threshold, based on the \textit{cumulative distribution}.
    
    \item \(f(u)\) reflects how likely it is to actually land on a specific value of \(u\) (from the \textit{probability density}).
\end{itemize}

Together, multiplying the \textit{CDF} and the \textit{PDF} allows you to combine:
\begin{itemize}
    \item The likelihood of being in the \textit{failure region} (via \(\pi(u)\)),
    \item And the likelihood of specific configurations (via \(f(u)\)).
\end{itemize}

This combined term $\hat{h}(\mathbf{u})$ is what you're sampling from in the Metropolis-Hastings algorithm.

\textbf{Why multiply, not add?}
\begin{itemize}
    \item \textit{Multiplying} the two terms gives you a \textbf{joint distribution} where both the \textit{likelihood} of the parameters and the \textit{probability of failure} are considered together.
    \item If you were to \textit{add} $\pi(u)$ and $f(u)$, you would be combining two very different types of quantities (one representing cumulative probability, the other representing point likelihood), which would not give you a valid combined distribution.
\end{itemize}

\textbf{Conclusion:}

It is correct and \textit{intuitive} to multiply the CDF ($\pi(u)$) and the PDF ($f(u)$) in this case, because:
\begin{itemize}
    \item You are combining the \textit{probability of being in the failure region} (captured by $\pi(u)$),
    \item And the \textit{likelihood of a particular sample} (captured by $f(u)$).
\end{itemize}

This forms a valid joint distribution that you can use in the Metropolis-Hastings algorithm.



% How the paper (Importance Sampling for PINNs) actually chooses points After identifying interesting regions (via DWT or another metric), the actual point selection is done using a weighted probability distribution over the sample pool: “We compute the sampling probability for each point using a normalized score... and sample points with probability proportional to that score.” Importance sampling for… In other words: Start with a pool of points (e.g. from LHS or uniformly sampled). Assign each point a score, like: Residual (if training a PINN), Gradient magnitude, Or your custom criterion (e.g. 𝜙(𝑥)=exp(−𝛼∣𝑔(𝑥)∣), to favor points near 𝑔=0).